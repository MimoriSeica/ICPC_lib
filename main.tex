\documentclass[10pt]{article}

\addtolength{\textwidth}{34mm}
\addtolength{\oddsidemargin}{-17mm}
\addtolength{\textheight}{34mm}
%\addtolength{\topmargin}{-17mm}
\addtolength{\topmargin}{-24mm}

\usepackage{listings,color}

%ここからソースコードの表示に関する設定
\definecolor{OliveGreen}{cmyk}{0.64,0,0.95,0.40}
\definecolor{colFunc}{rgb}{1,0.07,0.54}
\definecolor{CadetBlue}{cmyk}{0.62,0.57,0.23,0}
\definecolor{Brown}{cmyk}{0,0.81,1,0.60}
\definecolor{colID}{rgb}{0.63,0.44,0}
\lstset{
language=C++,%プログラミング言語によって変える。
basicstyle={\ttfamily\small},
keywordstyle={\color{OliveGreen}},
%\[2\]\[3\]はプログラミング言語によってあったり、なかったり
keywordstyle={[2]\color{colFunc}},
keywordstyle={[3]\color{CadetBlue}},%
commentstyle={\color{Brown}},
%identifierstyle={\color{colID}},
stringstyle=\color{blue},
tabsize=2,
frame=trBL,
numbers=left,
numberstyle={\ttfamily\small},
breaklines=true,%折り返し
captionpos=b
}
%ここまでソースコードの表示に関する設定

\title{Seica Lib}
\author{seica}
\date{2019/11/15}

\begin{document}

\maketitle

\tableofcontents
\clearpage

\section{Template}
\begin{lstlisting}
#include <bits/stdc++.h>

using namespace std;

#define REP(i,n) for(ll (i) = (0);(i) < (n);++i)
#define REV(i,n) for(ll (i) = (n) - 1;(i) >= 0;--i)
#define PB push_back
#define EB emplace_back
#define MP make_pair
#define FI first
#define SE second
#define SHOW1d(v,n) {REP(WW,n)cerr << v[WW] << ' ';cerr << endl << endl;}
#define SHOW2d(v,WW,HH) {REP(W_,WW){REP(H_,HH)cerr << v[W_][H_] << ' ';cerr << endl;}cerr << endl;}
#define ALL(v) v.begin(),v.end()
#define Decimal fixed<<setprecision(20)
#define INF 1000000000
#define LLINF 1000000000000000000LL
#define MOD 998244353

typedef long long ll;
typedef pair<ll,ll> P;

int main(){
	cin.tie(0);cout.tie(0);ios::sync_with_stdio(false);

	return 0;
}
\end{lstlisting}


\section{Graph}
\subsection{Dinic}
\begin{lstlisting}
struct Dinic {
	struct edge{int to;long long cap,rev;};
	vector<vector<edge>> G;
	vector<int> level;
	vector<int> iter;

	Dinic(int n):G(n, vector<edge>()),level(n),iter(n){}

	void add_edge(int from,int to,long long cap){
		G[from].push_back((edge){to,cap,(ll)G[to].size()});
		G[to].push_back((edge){from,0,(ll)G[from].size()-1});
	}

	void bfs(int s){
		fill(level.begin(), level.end(), -1);
		queue<int> que;
		level[s] = 0;
		que.push(s);
		while(!que.empty())	{
			int v = que.front();que.pop();
			for(int i = 0;i < G[v].size();++i){
				edge &e = G[v][i];
				if(e.cap > 0 && level[e.to] < 0){
					level[e.to] = level[v] + 1;
					que.push(e.to);
				}
			}
		}
	}

	int dfs(int v ,int t,long long f){
		if(v == t)return f;
		for(int &i = iter[v];i < G[v].size() ;++i){
			edge &e = G[v][i];
			if(e.cap > 0 && level[v] < level[e.to]){
				long long d = dfs(e.to,t,min(f,e.cap));
				if(d > 0){
					e.cap -= d;
					G[e.to][e.rev].cap += d;
					return d;
				}
			}
		}
		return 0;
	}

	long long solve(int s,int t){
		long long flow = 0;
		while(1){
			bfs(s);
			if(level[t] < 0)return flow;
			fill(iter.begin(), iter.end(), 0);
			long long f;while((f = dfs(s,t,INF)) > 0)flow += f;
		}
	}
};
\end{lstlisting}

\subsection{Min Cost Flow}
\begin{lstlisting}
class minCostFlow {

	struct edge { ll to, cap, cost, rev; };

	ll V;
	vector<vector<edge>> G;
	vector<ll> dist;
	vector<ll> prevv;
	vector<ll> preve;

public:

	minCostFlow(ll n): G(n), dist(n), prevv(n), preve(n), V(n){
	}

	void addEdge(ll from, ll to, ll cap, ll cost) {
		G[from].push_back((edge){to, cap, cost, (ll)G[to].size()});
		G[to].push_back((edge){from, 0, -cost, (ll)G[from].size() - 1});
	}

	ll solve(ll s, ll t, ll f) {
		ll ret = 0;
		while(f > 0) {
			fill(dist.begin(), dist.end(), LLINF);
			dist[s] = 0;
			bool update = true;
			while(update) {
				update = false;
				for(ll v = 0;v < V;v++) {
					if(dist[v] == LLINF)continue;
					for(ll i = 0;i < G[v].size();i++) {
						edge &e = G[v][i];
						if(e.cap > 0 && dist[e.to] > dist[v] + e.cost) {
							dist[e.to] = dist[v] + e.cost;
							prevv[e.to] = v;
							preve[e.to] = i;
							update = true;
						}
					}
				}
			}
			if(dist[t] == LLINF) {
				return LLINF;//流せない
			}

			ll d = f;
			for(ll v = t;v != s;v = prevv[v]) {
				d = min(d, G[prevv[v]][preve[v]].cap);
			}
			f -= d;
			ret += d * dist[t];
			for(ll v = t;v != s;v = prevv[v]) {
				edge &e = G[prevv[v]][preve[v]];
				e.cap -= d;
				G[v][e.rev].cap += d;
			}
		}
		return ret;
	}

};
\end{lstlisting}

\subsection{Max Two Matching}
\begin{lstlisting}
#define MAX_V 2222
vector<int> v[MAX_V];
int match[MAX_V];
bool used[MAX_V];

void add_edge(int a,int b){
	v[a].PB(b);
	v[b].PB(a);
}

bool matchdfs(int a){
	used[a] = true;
	for(int i = 0;i < v[a].size();i++)	{
		int u = v[a][i],w = match[u];
		if(w < 0 || !used[w] && matchdfs(w)){
			match[u] = a;
			match[a] = u;
			return true;
		}
	}
	return false;
}

//二部マッチング
int two_matching_max(int l){
	int res = 0;
	memset(match,-1,sizeof(match));
	REP(v,l){
		if(match[v] < 0){
			memset(used,0,sizeof(used));
			if(matchdfs(v))res++;
		}
	}
	return res;
}
\end{lstlisting}

\subsection{Strongly Connected Components}
\begin{lstlisting}
struct SCC {
	vector<vector<int>> G;
	vector<vector<int>> rG;
	vector<int> vs;
	vector<bool> used;
	vector<int> cmp;
	vector<int> sorted_num; // SCC 後のトポロジカルソートの順番が入る
	int sz,newsz;
	SCC(int n):G(n),rG(n),used(n,false),cmp(n){sz = n;}

	//辺を張る
	void add_edge(int from,int to){
		G[from].PB(to);
		rG[to].PB(from);
	}

	void dfs(int v){
		used[v] = true;
		REP(i,G[v].size()){
			if(!used[G[v][i]])dfs(G[v][i]);
		}
		vs.PB(v);
	}

	void rdfs(int v,int k){
		used[v] = false;
		cmp[v] = k;
		REP(i,rG[v].size()){
			if(used[rG[v][i]])rdfs(rG[v][i],k);
		}
	}

	//SCC実行
	int solve(){
		vs.clear();
		REP(i,sz)if(!used[i])dfs(i);
		int k = 0;
		for(int i = sz-1;i >= 0;i--){
			if(used[vs[i]]){
				rdfs(vs[i],k++);
			}
		}
		return newsz = k;
	}

	//SCC した後の頂点を返す
	int scc_node(int n){
		return cmp[n];
	}

	// 同じ頂点か
	bool isSameGroup(int a, int b){
		return cmp[a] == cmp[b];
	}

	// SCC 後のグラフのトポロジカルシート
	void topological_sort() {
		sorted_num.resize(newsz);
        vector<int> cou(newsz);
        vector<vector<int>> new_v(newsz);

        REP(i, sz){
            new_v[scc_node(i)].PB(i);
        }

        queue<int> q;
        REP(i, newsz){
            REP(j, new_v[i].size()){
                int node = new_v[i][j];
                REP(k, rG[node].size()){
                    int aite = rG[node][k];
                    if(isSameGroup(node, aite))continue;
                    cou[i]++;
                }
            }
            if(cou[i] == 0){
                q.push(i);
            }
        }

        int num = 0;
        while(!q.empty()){
            int now = q.front();q.pop();
            sorted_num[now] = num;
            num++;

            REP(i, new_v[now].size()){
                int node = new_v[now][i];
                REP(j, G[node].size()){
                    int aite = G[node][j];
                    if(isSameGroup(node, aite))continue;
                    cou[scc_node(aite)]--;

                    if(cou[scc_node(aite)] == 0){
                        q.push(scc_node(aite));
                    }
                }
            }
        }
	}

	// SCC 後のトポロジカル順序を返す
	int topological_order(int node){
		return sorted_num[cmp[node]];
	}
};
\end{lstlisting}


\section{String}
\subsection{Aho Corasick}
\begin{lstlisting}
template< int char_size >
struct TrieNode {
	int nxt[char_size + 1];

	int exist;
	vector< int > accept;

	TrieNode() : exist(0) {
		memset(nxt, -1, sizeof(nxt));
	}
};

template< int char_size, int margin >
struct Trie {
	using Node = TrieNode< char_size >;

	vector< Node > nodes;
	int root;

	Trie() : root(0) {
		nodes.push_back(Node());
	}

	void update_direct(int node, int id) {
		nodes[node].accept.push_back(id);
	}

	void update_child(int node, int child, int id) {
		++nodes[node].exist;
	}

	void add(const string &str, int str_index, int node_index, int id) {
		if(str_index == str.size()) {
			update_direct(node_index, id);
		} else {
			const int c = str[str_index] - margin;
			if(nodes[node_index].nxt[c] == -1) {
				nodes[node_index].nxt[c] = (int) nodes.size();
				nodes.push_back(Node());
			}
			add(str, str_index + 1, nodes[node_index].nxt[c], id);
			update_child(node_index, nodes[node_index].nxt[c], id);
		}
	}

	void add(const string &str, int id) {
		add(str, 0, 0, id);
	}

	void add(const string &str) {
		add(str, nodes[0].exist);
	}

	void query(const string &str, const function< void(int) > &f, int str_index, int node_index) {
		for(auto &idx : nodes[node_index].accept) f(idx);
		if(str_index == str.size()) {
			return;
		} else {
			const int c = str[str_index] - margin;
			if(nodes[node_index].nxt[c] == -1) return;
			query(str, f, str_index + 1, nodes[node_index].nxt[c]);
		}
	}

	void query(const string &str, const function< void(int) > &f) {
		query(str, f, 0, 0);
	}

	int size() const {
		return (nodes[0].exist);
	}

	int nodesize() const {
		return ((int) nodes.size());
	}
};


template< int char_size, int margin >
struct AhoCorasick : Trie< char_size, margin > {
	using TRIE = Trie< char_size, margin >;
	static const int FAIL = char_size;
	vector< int > correct;

	AhoCorasick() : TRIE() {}

	void build(bool heavy = true) {
		correct.resize(TRIE::nodes.size());
		for(int i = 0; i < TRIE::nodes.size(); i++) {
			correct[i] = (int) TRIE::nodes[i].accept.size();
		}

		queue< int > que;
		for(int i = 0; i <= char_size; i++) {
			if(~TRIE::nodes[0].nxt[i]) {
				TRIE::nodes[TRIE::nodes[0].nxt[i]].nxt[FAIL] = 0;
				que.emplace(TRIE::nodes[0].nxt[i]);
			} else {
				TRIE::nodes[0].nxt[i] = 0;
			}
		}
		while(!que.empty()) {
			TrieNode< char_size > &now = TRIE::nodes[que.front()];
			correct[que.front()] += correct[now.nxt[FAIL]];
			que.pop();
			for(int i = 0; i < char_size; i++) {
				if(now.nxt[i] == -1) continue;
				int fail = now.nxt[FAIL];
				while(TRIE::nodes[fail].nxt[i] == -1) fail = TRIE::nodes[fail].nxt[FAIL];
				TRIE::nodes[now.nxt[i]].nxt[FAIL] = TRIE::nodes[fail].nxt[i];
				if(heavy) {
					auto &u = TRIE::nodes[now.nxt[i]].accept;
					auto &v = TRIE::nodes[TRIE::nodes[fail].nxt[i]].accept;
					vector< int > accept;
					set_union(begin(u), end(u), begin(v), end(v), back_inserter(accept));
					u = accept;
				}
				que.emplace(now.nxt[i]);
			}

		}
	}

	int match(const string &str, vector< int > &result, int now = 0) {
		result.assign(TRIE::size(), 0);
		int count = 0;
		for(auto &c : str) {
			while(TRIE::nodes[now].nxt[c - margin] == -1) now = TRIE::nodes[now].nxt[FAIL];
			now = TRIE::nodes[now].nxt[c - margin];
			count += correct[now];
			for(auto &v : TRIE::nodes[now].accept) ++result[v];
		}
		return (count);
	}

	int match(const string &str, vector< vector< int > > &result, int now = 0) {
		result.assign(TRIE::size(), vector<int>());
		int count = 0;
		for(int i = 0;i < str.size();i++) {
			char c = str[i];
			while(TRIE::nodes[now].nxt[c - margin] == -1) now = TRIE::nodes[now].nxt[FAIL];
			now = TRIE::nodes[now].nxt[c - margin];
			count += correct[now];
			for(auto &v : TRIE::nodes[now].accept) result[v].PB(i);
		}
		return (count);
	}

	int move(const char &c, int &now) {
		int count = 0;
		while(TRIE::nodes[now].nxt[c - margin] == -1) now = TRIE::nodes[now].nxt[FAIL];
		now = TRIE::nodes[now].nxt[c - margin];
		count += correct[now];
		return (count);
	}
};

int needs[111111];
string in[111111];
//https://codeforces.com/problemset/problem/963/D
int main(){

	fastcin();
	AhoCorasick<26, 'a'> aho;

	string str;cin >> str;
	int n;cin >> n;
	REP(i,n){
		cin >> needs[i];
		cin >> in[i];
		aho.add(in[i]);
	}
	aho.build();
	vector<vector<int>> ans(n);
	aho.match(str, ans);

	REP(i,n){
		if(ans[i].size() < needs[i]){
			cout << -1 << endl;
			continue;
		}

		int out = INF;
		REP(j,ans[i].size() - needs[i] + 1){
			out = min(out, ans[i][j+needs[i]-1] - ans[i][j] + (int)in[i].size());
		}
		cout << out << endl;
	}

	return 0;
}
\end{lstlisting}

\subsection{Z algorithm}
\begin{lstlisting}
vector<int> z_algorithm(string s){
   int n = s.size();
   vector<int> ret(n);
   ret[0] = n;
   int i = 1, j = 0;
   while(i < s.size()){
       while(i+j < s.size() && s[j] == s[i+j])++j;
       ret[i] = j;
       if(j == 0){ ++i; continue; }
       int k = 1;
       while(i + k < s.size() && k + ret[k] < j)ret[i+k] = ret[k],++k;
       i += k; j -= k;
   }
   return ret;
}
\end{lstlisting}

\subsection{Rolling Hash}
\begin{lstlisting}
#define N 3
ll mod[] = {999999937,1000000007,1000000009};
ll kake = 123123;

ll m_pow(ll a,ll b,ll p){
	ll ret = 1;
	ll tmp = a;
	for(ll i = 0LL;i < 60;i++){
		if(b & (1LL << i)){
			ret *= tmp;
			ret %= p;
		}
		tmp *= tmp;
		tmp %= p;
	}
	return ret;
}

//strが元の文字列、sが探したい文字列、一致する頭のリストを返す、O(N+M)
vector<int> RollingHash(string str,string s){
	vector<int> ret;
	ll mark[N] = {};
	REP(i,s.size()){
		REP(j,N){
			mark[j] *= kake;
			mark[j] += (int)s[i];
			mark[j] %= mod[j];
		}
	}
	ll po[N];
	REP(i,N)po[i] = m_pow(kake,s.size()-1,mod[i]);
	ll now[N] = {};
	REP(i,str.size()){
		bool flag = true;
		REP(j,N){
			if(i >= s.size()){
				ll tmp = po[j] * (int)str[i-s.size()] % mod[j];
				now[j] = (now[j] + mod[j] - tmp) % mod[j];
			}
			now[j] *= kake;
			now[j] += (int)str[i];
			now[j] %= mod[j];
			if(now[j] != mark[j])flag = false;
		}
		if(flag)ret.PB(i-s.size()+1);
	}
	return ret;
}

int main(){

	string a,b;cin >> a >> b;
	vector<int> ans = RollingHash(a,b);
	REP(i,ans.size()){
		cout << ans[i] << endl;
	}

	return 0;
}
\end{lstlisting}

\subsection{Suffix Array}
\begin{lstlisting}
struct SuffixArray {

	int n,k;
	string S;
	vector<int> rnk,tmp,sa,lcp;

	SuffixArray(string s){
		S = s;
		n = s.size();
		rnk.resize(n+1);
		tmp.resize(n+1);
		sa.resize(n+1);
		lcp.resize(n+1);
		construct_sa();
		construct_lcp();
	}

	void construct_sa(){

		for(int i = 0; i <= n;i++){
			sa[i] = i;
			rnk[i] = i < n ? S[i] : -1;
		}

		auto comp = [&](int &i, int &j){
			if(rnk[i] != rnk[j]){
				return rnk[i] < rnk[j];
			}
			else{
				int ri =( i + k <= n )? rnk[i + k] : -1;
				int rj =( j + k <= n )? rnk[j + k] : -1;
				return ri < rj;
			}
		};

		for(k = 1;k <= n ; k *= 2){
			sort(sa.begin(), sa.end(), comp);

			tmp[sa[0]] = 0;
			for(int i = 1; i <= n; i++)	{
				tmp[sa[i]] = tmp[sa[i -1]] + (comp(sa[i - 1],sa[i]) ? 1 : 0);
			}
			for(int i = 0;i <= n;i++){
				rnk[i] = tmp[i];
			}
		}
	}

	bool contain(string t){
		int a = 0,b = S.length();
		while(b - a > 1){
			int c = (a + b) / 2;

			if(S.compare(sa[c],t.length(),t) < 0) a = c;
			else b = c;
		}

		return (S.compare(sa[b], t.length(),t) == 0);
	}

	void construct_lcp(){
		int n = S.length();
		for(int i = 0;i <= n;i++)rnk[sa[i]] = i;

		int h = 0;
		lcp[0] = 0;
		for(int i = 0;i < n;i++){
			int j = sa[rnk[i] - 1];

			if(h > 0)h--;
			for(;j + h < n && i + h < n;h++){
				if(S[j + h] != S[i + h])break;
			}
			lcp[rnk[i] - 1] = h;
		}
	}
};

int main()
{
	string S;
	int N;

	cin >> S;
	cin >> N;
	SuffixArray sa(S);

	for(int i = 0 ;i < N;i++)
	{
		string T;
		cin >> T;
		if(sa.contain(T)) cout << 1 << endl;
		else cout << 0 << endl;
	}

	return 0;
}
\end{lstlisting}


\section{Data Structure}
\subsection{Union Find}
\begin{lstlisting}
struct UF
{
	vector<int> par; // 親のインデックスを記憶する配列
	vector<int> sz; // サイズを記憶する。
	vector<int> rank;
	// 初期化
	UF(int n):par(n),sz(n),rank(n){
		for(int i = 0; i < n; i++){
			par[i] = i;sz[i] = 1;rank[i] = 0;
		}
	}
	// 親を求める
	int find(int x) {
		if (par[x] == x) return x;
		else return par[x] = find(par[x]);
	}
	// xとyの属する集合を併合
	void unite(int x, int y) {
		x = find(x); y = find(y);
		if (x == y) return;
		if(rank[x] < rank[y]){
			par[x] = y;
			sz[y] += sz[x];
		}
		else{
			par[y] = x;
			sz[x] += sz[y];
			if(rank[x] == rank[y])rank[x]++;
		}
	}
	// xとyが同じ集合ならtrue
	bool same(int x, int y) { return find(x) == find(y); }
	// 素の集合のサイズを求める
	int size(int n){return sz[find(n)];}
};
\end{lstlisting}

\subsection{Segment Tree}
\begin{lstlisting}
/*
0-index
functionの中を書いて使う
segtree<ll> seg(n, [](ll a, ll b){return a + b;}, 0);
*/
template<typename T>
struct segtree{
	int n;
	vector<T> dat;
	T ZERO;
	function<T(T,T)> func;//要素と要素をマージする関数

	segtree(int n_, function<T(T,T)> func_, T zero):func(func_), ZERO(zero){
	  n = 1;
	  while(n < n_)n*=2;
	  dat.resize(2*n);
	  for(int i = 0;i < 2*n;i++)dat[i] = ZERO;
	}

	void update(int k, T a){
	  k += n - 1;dat[k] = a;
	  while(k > 0){
		k = (k-1)/2;
		dat[k] = func(dat[k*2+1],dat[k*2+2]);
	  }
	}

	void set(int k, T a) {
		k += n - 1;dat[k] = a;
	}

	void build(){
		for(int i = n - 2;i >= 0;i--){
			dat[i] = func(dat[2 * i + 1], dat[2 * i + 2]);
		}
	}

	T get(int k){return dat[k+n-1];}

	T que(int a,int b){
		T L = ZERO,R = ZERO;
		int A = a + n - 1;
		int B = b + n - 1;
		while(A < B){
			if((A & 1) == 0)L = func(L,dat[A++]);
            if((B & 1) == 0)R = func(dat[--B],R);
            A = A >> 1;
            B = B >> 1;
		}
		return func(L,R);
	}
};
\end{lstlisting}

\subsection{Lazy Segment Tree}
\begin{lstlisting}
template <typename T,typename E>
struct SegmentTree{
	typedef function<T(T,T)> F;
	typedef function<T(T,E)> G;
	typedef function<E(E,E)> H;
	typedef function<E(E,int)> P;
	int n;
	F f;
	G g;
	H h;
	P p;
	T d1;
	E d0;
	vector<T> dat;
	vector<E> laz;

	SegmentTree(int n_,
				F f,//要素と要素をマージする関数
				G g,//要素に作用素を作用させる関数
				H h,//作用素と作用素をマージする関数
				T d1,//要素のモノイド
				E d0,//作用素のモノイド
				vector<T> v=vector<T>(),
				P p=[](E a,int b){return a;})://区間に対する操作が要素数に比例して変化する場合
	f(f),g(g),h(h),d1(d1),d0(d0),p(p){
		init(n_);
		if(n_==(int)v.size()) build(n_,v);
	}

	void init(int n_){
		n=1;
		while(n<n_) n*=2;
		dat.clear();
		dat.resize(2*n-1,d1);
		laz.clear();
		laz.resize(2*n-1,d0);
	}

	void build(int n_, vector<T> v){
		for(int i=0;i<n_;i++) dat[i+n-1]=v[i];
		for(int i=n-2;i>=0;i--){
			dat[i]=f(dat[i*2+1],dat[i*2+2]);
		}
	}

	inline void eval(int len,int k){
		if(laz[k]==d0) return;
		if(k*2+1<n*2-1){
			laz[k*2+1]=h(laz[k*2+1],laz[k]);
			laz[k*2+2]=h(laz[k*2+2],laz[k]);
		}
		dat[k]=g(dat[k],p(laz[k],len));
		laz[k]=d0;
	}


	T update(int a,int b,E x,int k,int l,int r){
		eval(r-l,k);
		if(r<=a||b<=l) return dat[k];
		if(a<=l&&r<=b){
			laz[k]=h(laz[k],x);
			return g(dat[k],p(laz[k],r-l));
		}
		return dat[k]=f(update(a,b,x,k*2+1,l,(l+r)/2),
				update(a,b,x,k*2+2,(l+r)/2,r));
	}

	T update(int a,int b,E x){
		return update(a,b,x,0,0,n);
	}

	T query(int a,int b,int k,int l,int r){
		eval(r-l,k);
		if(r<=a||b<=l) return d1;
		if(a<=l&&r<=b) return dat[k];
		T vl=query(a,b,k*2+1,l,(l+r)/2);
		T vr=query(a,b,k*2+2,(l+r)/2,r);
		return f(vl,vr);
	}

	T query(int a,int b){
		return query(a,b,0,0,n);
	}
};
/**
	SegmentTree(int n_,
				F f,//要素と要素をマージする関数
				G g,//要素に作用素を作用させる関数
				H h,//作用素と作用素をマージする関数
				T d1,//要素のモノイド
				E d0,//作用素のモノイド
				vector<T> v=vector<T>(),//初期配列
				P p=[](E a,int b){return a;})://区間に対する操作が要素数に比例して変化する場合

	区間加算、区間和
	SegmentTree<ll,ll> seg(n,
	[](ll a,ll b){return a + b;},
	[](ll a,ll b){return a + b;},
	[](ll a,ll b){return a + b;},
	0,
	0,
	vector<ll>(n,0),
	[](ll a,int b){return a * b;});

	区間更新、区間最小
	SegmentTree<ll,ll> seg(n,
	[](ll a,ll b){return min(a, b);},
	[](ll a,ll b){return b;},
	[](ll a,ll b){return b;},
	INF,
	INF,
	vector<ll>(n,INF),
	[](ll a,int b){return a;});

	区間加算、区間最小

	SegmentTree<ll,ll> seg(n,
	[](ll a,ll b){return min(a, b);},
	[](ll a,ll b){return a + b;},
	[](ll a,ll b){return a + b;},
	INF,
	0,
	vector<ll>(n,0),
	[](ll a,int b){return a;});

	区間更新、区間和
	SegmentTree<ll,ll> seg(n,
	[](ll a,ll b){return a + b;},
	[](ll a,ll b){return b;},
	[](ll a,ll b){return b;},
	0,
	INF,
	vector<ll>(n,0),
	[](ll a,int b){return a * b;});
**/
\end{lstlisting}

\subsection{Lowest Common Ancestor}
\begin{lstlisting}
//普通のLCA(UFと一緒に)
struct LCA{
	//隣接リスト、(相手、重み)
	vector<vector<pair<int,int> > > g;
	//親
	vector<vector<int> > parent;
	//深さ
	vector<int> depth;
	UF uf;
	int sz;
	vector<bool> inited;

	LCA(int n):g(n),parent(20,vector<int>(n)),
	inited(n,false),depth(n),uf(n){sz = n;}

	//辺の追加(点、点、重み)
	void add_edge(int u,int v,int w = 1){
		if(uf.same(u,v))return;
		g[v].PB(MP(u,w));
		g[u].PB(MP(v,w));
		uf.unite(v,u);
	}

	void dfs(int v,int p,int d){
		inited[v] = true;
		parent[0][v] = p;
		depth[v] = d;
		REP(i,g[v].size()){
			if(g[v][i].FI != p)dfs(g[v][i].FI,v,d+1);
		}
	}

	void init(){
		REP(i,sz)if(!inited[i])dfs(uf.find(i),-1,0);
		REP(k,19){
			REP(i,sz){
				if(parent[k][i] < 0)parent[k+1][i] = -1;
				else parent[k+1][i] = parent[k][parent[k][i]];
			}
		}
	}

	int ret_lca(int a,int b){

		if(!uf.same(a,b))return -1;

		if(depth[a] > depth[b])swap(a,b);
		REP(k,20){
			if((depth[b] - depth[a])>>k&1)b = parent[k][b];
		}
		if(a == b)return a;

		for(int k = 19;k >= 0;k--){
			if(parent[k][a] != parent[k][b])
			{
				a = parent[k][a];
				b = parent[k][b];
			}
		}
		return parent[0][a];
	}

};

//任意の二点のパスの中の最大値を求められる。(UFと一緒に)
struct LCA{
	//隣接リスト、(相手、重み)
	vector<vector<pair<int,int> > > g;
	//親
	vector<vector<int> > parent;
	vector<vector<int> > value;
	//深さ
	vector<int> depth;
	UF uf;
	int sz;
	vector<bool> inited;

	map<pair<int,int> ,int> mp;

	pair<int,int> Pair(int a,int b){
		return MP(min(a,b),max(a,b));
	}

	LCA(int n):g(n),parent(20,vector<int>(n)),
	inited(n,false),depth(n),uf(n),value(20,vector<int>(n)){sz = n;}

	//辺の追加(点、点、重み)
	void add_edge(int u,int v,int w = 1){
		if(uf.same(u,v))return;
		g[v].PB(MP(u,w));
		g[u].PB(MP(v,w));
		uf.unite(v,u);
		mp[Pair(u,v)] = w;
	}

	void dfs(int v,int p,int d){
		inited[v] = true;
		value[0][v] = mp[Pair(v,p)];
		parent[0][v] = p;
		depth[v] = d;
		REP(i,g[v].size())
		{
			if(g[v][i].FI != p)dfs(g[v][i].FI,v,d+1);
		}
	}

	//頂点の追加後、ここで構築する
	void init(){
		REP(i,sz)if(!inited[i])dfs(uf.find(i),-1,0);
		REP(k,19){
			REP(i,sz){
				if(parent[k][i] < 0){
					parent[k+1][i] = -1;
					value[k+1][i] = value[k][i];
				}
				else {
					parent[k+1][i] = parent[k][parent[k][i]];
					value[k+1][i] = max(value[k][i],value[k][parent[k][i]]);
				}
			}
		}
	}

	int ret_lca(int a,int b){

		if(!uf.same(a,b))return -1;

		if(depth[a] > depth[b])swap(a,b);
		REP(k,20){
			if((depth[b] - depth[a])>>k&1)b = parent[k][b];
		}
		if(a == b)return a;

		for(int k = 19;k >= 0;k--){
			if(parent[k][a] != parent[k][b]){
				a = parent[k][a];
				b = parent[k][b];
			}
		}
		return parent[0][a];
	}

	int ret_value(int a,int b){

		if(!uf.same(a,b))return -1;
		int ret = 0;
		if(depth[a] > depth[b])swap(a,b);
		REP(k,20){
			if((depth[b] - depth[a])>>k&1){
				ret = max(ret,value[k][b]);
				b = parent[k][b];
			}
		}
		if(a == b)return ret;

		for(int k = 19;k >= 0;k--){
			if(parent[k][a] != parent[k][b]){
				ret = max(ret,value[k][a]);
				ret = max(ret,value[k][b]);
				a = parent[k][a];
				b = parent[k][b];
			}
		}
		ret = max(ret,value[0][a]);
		ret = max(ret,value[0][b]);

		return ret;
	}

};

int main(){
	return 0;
}
\end{lstlisting}

\subsection{Convex Hull Trick}
\begin{lstlisting}
/*
評価する直線の傾きが単調減少
求めるxが単調増加の時の最小値
*/
class ConvexHullTrick_Monotony{
	const ll inf = 1e9;
	struct Line{
		ll a,b;
		Line(ll x,ll y):a(x),b(y){};
	};

	deque<Line> deq;

public:
	ConvexHullTrick_Monotony(){}
	void Add_back(ll a,ll b){
		if(deq.size() <= 1){
			deq.push_back(Line(a,b));
		}
		else{
			while(deq.size() > 1){
				ll a1 = deq[deq.size()-2].a;
				ll b1 = deq[deq.size()-2].b;
				ll a2 = deq[deq.size()-1].a;
				ll b2 = deq[deq.size()-1].b;
				if((b2 - b1)*(a2 - a) < (b - b2)*(a1 - a2)){
					break;
				}
				else{
					deq.pop_back();
				}
			}
			deq.push_back(Line(a,b));
		}
	}

	ll seek_front(ll x){
		while(deq.size() > 1 && deq[0].a * x + deq[0].b > deq[1].a * x + deq[1].b){
			deq.pop_front();
		}
		return deq[0].a * x + deq[0].b;
	}
};

int main(){
	ConvexHullTrick_Monotony CHT;

	int n;cin >> n;

	REP(i,n){
		ll a;cin >> a;
		CHT.Add_back(-2*(i+1),a+(i+1)*(i+1));
	}

	REP(i,n){
		cout << CHT.seek_front(i+1) + (i+1)*(i+1) << endl;
	}

	return 0;
}
\end{lstlisting}

\subsection{Li Chao Tree}
\begin{lstlisting}
template<typename T>
struct LiChaoTree {

    vector<pair<T, T>> dat;
    vector<ll> x;
    pair<T, T> inf;
    int n;

    LiChaoTree(int n_, pair<T, T> inf_) {
        n = 1;
        while(n < n_ + 1)n *= 2;
        inf = inf_;

        dat = vector<pair<T, T>>(2 * n, inf);
        x = vector<ll>(n + 1);
        REP(i, x.size())x[i] = i;
    }

    LiChaoTree(vector<ll> coordinate, pair<T, T> inf_) {
        n = 1;
        while(n < coordinate.size() + 1)n *= 2;
        inf = inf_;

        dat = vector<pair<T, T>>(2 * n, inf);
        x = vector<ll>(n + 1);
		REP(i, x.size()){
            if(i < coordinate.size())x[i] = coordinate[i];
            else x[i] = x[i-1] + 1;
        }
    }

    T func(pair<T, T> line, int a){
        return line.first * a + line.second;
    }

    void add_line(pair<T, T> line, int k, int l, int r) {
        int m = (l + r) / 2;
        T lx = x[l];
        T mx = x[m];
        T rx = x[r];
        bool left = (func(line, lx) < func(dat[k], lx));
        bool mid = (func(line, mx) < func(dat[k], mx));
        bool right = (func(line, rx) < func(dat[k], rx));

        if(!left && !right){
            return;
        }
        if(mid){
            swap(line, dat[k]);
        }
        if(left != mid && 2*k+1 < dat.size()){
            add_line(line, 2*k+1, l, m);
        }
        else if(right != mid && 2*k+2 < dat.size()){
            add_line(line, 2*k+2, m, r);
        }
    }

    // [a, b) に線分を追加
    void add_line(pair<T, T> line, int a, int b) {
        int A = a + n - 1;
        int B = b + n - 1;
        int sz = 1;
        int al = a;
        int bl = b;
        while(A < B){
            if((A & 1) == 0){
                add_line(line, A++, al, al+sz);
                al += sz;
            }
            if((B & 1) == 0){
                bl -= sz;
                add_line(line, --B, bl, bl+sz);
            }
            A = A >> 1;
            B = B >> 1;
            sz = sz << 1;
        }
    }

    // 直線を追加
    void add_line(pair<T, T> line){
        add_line(line, 0, 0, n);
    }

    T query(int a){
        int ax = x[a];
        a += n - 1;
        T ans = func(dat[a], ax);
        while(a > 0){
            a = (a - 1) / 2;
            ans = min(ans, func(dat[a], ax));
        }
        return ans;
    }

};
\end{lstlisting}


\section{Geometry}
\begin{lstlisting}
#define curr(PP, i) PP[i]
#define next(PP, i) PP[(i+1)%PP.size()]
#define diff(PP, i) (next(PP, i) - curr(PP, i))
#define eq(n,m) (abs((n)-(m)) < EPS)

typedef long long ll;
typedef pair<ll, ll> P;

const double EPS = 1e-8;
const double EPS_GIG = 1e-3;
const double PI = acos(-1.0);
typedef complex<double> point;
namespace std {
	bool operator < (const point& a, const point& b) {
		return real(a) != real(b) ? real(a) < real(b) : imag(a) < imag(b);
	}

	bool operator == (const point& a,const point& b) {
		return (abs(a.real() - b.real()) < EPS && abs(a.imag() - b.imag()) < EPS);
	}
}
struct circle {
	point p; double r;
	circle(){}
	circle(const point &p, double r) : p(p), r(r) { }
};

// 扇型、中心と半径、二つの端点
// 現在中心角が180未満の前提
struct sector {
	point o;
	point a, b;
	double r;
	sector(){}
	sector(point O, point A, point B, double _r) :o(O), a(A), b(B), r(_r) {}
};

struct segment : public array<point, 2> {
	segment(const point &a, const point &b) {
		at(0) = a;
		at(1) = b;
	}
};

double cross(const point& a, const point& b) {
	return imag(conj(a)*b);
}

double dot(const point& a, const point& b) {
	return real(conj(a)*b);
}

//角度足し算
double add_rad(double a,double b){
	double ret = a + b;
	if(ret > 2 * PI)ret -= 2 * PI;
	return ret;
}

//なす角(vector)
double angle(const point &a,const point &b) {
	auto tmp = abs(arg(a) - arg(b));
	return min(tmp, 2 * PI - tmp);
}

double angle(const segment &s1,const segment &s2) {
	return angle(s1[1] - s1[0], s2[1] - s2[0]);
}

//点の回転
point rotate(const point &p, double rad) {
	double x = p.real() * cos(rad) - p.imag() * sin(rad);
	double y = p.imag() * cos(rad) + p.real() * sin(rad);
	return point(x, y);
}

//並行
bool isParallel(const point &a, const point &b){
    return abs(cross(a,b)) < EPS;
}
bool isParallel(const segment &a, const segment &b){
    return isParallel(a[1]-a[0], b[1]-b[0]);
}
//直行
bool isOrthogonal(const point &a,const point &b){
	return abs(angle(a,b) - PI / 2) < EPS;
}
bool isOrthogonal(const segment &a,const segment &b){
	return isOrthogonal(a[1]-a[0],b[1]-b[0]);
}

/*
a → b で時計方向に折れて b → c
a → b で半時計方向に折れて b → c
a → b で逆を向いて a を通り越して b → c
a → b でそのまま b → c
a → b で逆を向いて b → c ( または b == c )
*/

int ccw(point a, point b, point c) {
	b -= a; c -= a;
	if (cross(b, c) > EPS)   return +1;       // counter clockwise
	if (cross(b, c) + EPS < 0)   return -1;       // clockwise
	if (dot(b, c) < 0)     return +2;       // c--a--b on line
	if (norm(b) < norm(c)) return -2;       // a--b--c on line
	return 0;
}

bool intersectLL(const segment &l, const segment &m) {
	return abs(cross(l[1] - l[0], m[1] - m[0])) > EPS || // non-parallel
		abs(cross(l[1] - l[0], m[0] - l[0])) < EPS;   // same line
}
bool intersectLS(const segment &l, const segment &s) {
	return cross(l[1] - l[0], s[0] - l[0])*       // s[0] is left of l
		cross(l[1] - l[0], s[1] - l[0]) < EPS; // s[1] is right of l
}
bool intersectLP(const segment &l, const point &p) {
	return abs(cross(l[1] - p, l[0] - p)) < EPS;
}
bool intersectSP(const segment &s, const point &p) {
	auto a = s[0] - p;
	auto b = s[1] - p;
	return (abs(cross(a, b)) < EPS && dot(a, b) <= EPS); // triangle inequality
}
//端点の交差も考える
bool intersectSS(const segment &s, const segment &t) {
	return ccw(s[0], s[1], t[0]) * ccw(s[0], s[1], t[1]) <= 0 &&
		ccw(t[0], t[1], s[0]) * ccw(t[0], t[1], s[1]) <= 0;
}
//端点の交差hは考えない
bool strictIntersectSS(const segment &s, const segment &t) {
	return ccw(s[0], s[1], t[0]) * ccw(s[0], s[1], t[1]) == -1 &&
		ccw(t[0], t[1], s[0]) * ccw(t[0], t[1], s[1]) == -1;
}

point projection(const segment &l, const point &p) {
	double t = dot(p - l[0], l[0] - l[1]) / norm(l[0] - l[1]);
	return l[0] + t*(l[0] - l[1]);
}
point reflection(const segment &l, const point &p) {
	return p + 2. * (projection(l, p) - p);
}
double distanceLP(const segment &l, const point &p) {
	return abs(p - projection(l, p));
}
double distanceLL(const segment &l, const segment &m) {
	return intersectLL(l, m) ? 0 : distanceLP(l, m[0]);
}
double distanceLS(const segment &l, const segment &s) {
	if (intersectLS(l, s)) return 0;
	return min(distanceLP(l, s[0]), distanceLP(l, s[1]));
}
double distanceSP(const segment &s, const point &p) {
	const point r = projection(s, p);
	if (intersectSP(s, r)) return abs(r - p);
	return min(abs(s[0] - p), abs(s[1] - p));
}
double distanceSS(const segment &s, const segment &t) {
	if (intersectSS(s, t)) return 0;
	return min(min(distanceSP(s, t[0]), distanceSP(s, t[1])),
		min(distanceSP(t, s[0]), distanceSP(t, s[1])));
}
double distancePP(const point &a,const point &b){
	return abs(a-b);
}

/*多角形内包判定
half-line crossing method
OUT:0
ON:1
IN:2
*/
int contains(const vector<point>& Poly, const point& p) {
	bool in = false;
	for (int i = 0; i < Poly.size(); ++i) {
		point a = curr(Poly, i) - p, b = next(Poly, i) - p;
		if (imag(a) > imag(b)) swap(a, b);
		if (imag(a) + EPS <= 0 && EPS < imag(b))
			if (cross(a, b) < 0) in = !in;
		if (abs(cross(a, b)) < EPS && dot(a, b) <= EPS) return 1;
	}
	return in ? 2 : 0;
}

/*
OUT:0
ON:1
IN:2
*/
int contain_sector(const sector &sec, point &p){
	if(eq(abs(p - sec.o), sec.r))return 1;
	if(intersectSP(segment(sec.o, sec.a), p))return 1;
	if(intersectSP(segment(sec.o, sec.b), p))return 1;
	if(abs(p - sec.o) + EPS > sec.r)return 0;
	point vec = p - sec.o;
	point vecA = sec.a - sec.o;
	point vecB = sec.b - sec.o;
	if(angle(vec, vecA) + EPS < angle(vecA, vecB) && angle(vec, vecB) + EPS < angle(vecA, vecB))return 2;
	return 0;
}

//交点
point crosspointSS(const segment &l, const segment &m) {
	double A = cross(l[1] - l[0], m[1] - m[0]);
	double B = cross(l[1] - l[0], l[1] - m[0]);
	if (abs(A) < EPS && abs(B) < EPS) return m[0]; // same line
	if (abs(A) < EPS) return point(INF,INF); // !!!PRECONDITION NOT SATISFIED!!!
	return m[0] + B / A * (m[1] - m[0]);
}

vector<point> crosspointCL(const circle &c, const segment &l) {
	auto ret = vector<point>(2, point(INF, INF));
	auto pro_p = projection(l, c.p);
	auto dist = distanceLP(l, c.p);
	if(abs(dist - c.r) < EPS){
		ret[0] = pro_p;
		return ret;
	}
	if(c.r < dist){
		return ret;
	}
	point vec = (l[1] - l[0]) * sqrt(c.r * c.r - dist * dist) / abs(l[1] - l[0]);
	ret[0] = pro_p + vec;
	ret[1] = pro_p - vec;
	return ret;
}

vector<point> crosspointCC(const circle c1, const circle c2) {
	auto ret = vector<point>(2, point(INF, INF));
	auto dist = abs(c2.p - c1.p);
	if(eq(dist, c1.r + c2.r) || eq(dist, abs(c2.r - c1.r))){
		auto tmp = c2.p - c1.p;
		ret[0] = c1.p + tmp * (c1.r / dist);
		return ret;
	}
	if(c1.r + c2.r < dist || dist < abs(c1.r - c2.r)){
		return ret;
	}
	auto alpha = acos((c1.r * c1.r + dist * dist - c2.r * c2.r) / (2 * c1.r * dist));
	auto theta = atan2(c2.p.imag() - c1.p.imag(), c2.p.real() - c1.p.real());
	ret[0] = c1.p + point(cos(theta + alpha) * c1.r, sin(theta + alpha) * c1.r);
	ret[1] = c1.p + point(cos(theta - alpha) * c1.r, sin(theta - alpha) * c1.r);
	return ret;
}

bool isOnSector(const sector sec, const point p) {
	point vec = p - sec.o;
	point vecA = sec.a - sec.o;
	point vecB = sec.b - sec.o;
	if(eq(angle(vec, vecA) + angle(vec, vecB), angle(vecA, vecB)))return true;
	return false;
}

vector<point> crosspointSecS(const sector sec, const segment s) {
	circle c = circle(sec.o, sec.r);
	auto ret = crosspointCL(c, s);
	point inf = point(INF, INF);
	REP(i, 2){
		if(eq(ret[i], inf))continue;
		if(!isOnSector(sec, ret[i])){
			ret[i] = inf;
			continue;
		}
		if(!intersectSP(s, ret[i])){
			ret[i] = inf;
		}
	}
	return ret;
}
vector<point> crosspointSecSec(const sector sec1, const sector sec2) {
	circle c1 = circle(sec1.o, sec1.r);
	circle c2 = circle(sec2.o, sec2.r);
	auto ret = crosspointCC(c1, c2);
	point inf = point(INF, INF);
	REP(i, 2){
		if(!isOnSector(sec1, ret[i])){
			ret[i] = inf;
			continue;
		}
		if(!isOnSector(sec2, ret[i])){
			ret[i] = inf;
		}
	}
	return ret;
}


//凸包
vector<point> convex_hull(vector<point> ps) {
	int n = ps.size(), k = 0;
	sort(ps.begin(), ps.end());
	vector<point> ch(2*n);
	for (int i = 0; i < n; ch[k++] = ps[i++]) // lower-hull
		while (k >= 2 && ccw(ch[k-2], ch[k-1], ps[i]) == -1) --k;
	for (int i = n-2, t = k+1;i >= 0; ch[k++] = ps[i--]) // upper-hull
		while (k >= t && ccw(ch[k-2], ch[k-1], ps[i]) == -1) --k;
	ch.resize(k - 1);
	return ch;
}

//見えるか(可視グラフ用)
bool block_off(const point &a, const point &b, const vector<point> &obj) {
  point m = (a + b) * 0.5;
  bool on = false, in = false;
  for (int j = 0; j < obj.size(); ++j) {
    point c = curr(obj,j), d = next(obj,j);
    if (imag(d) < imag(c)) swap(c, d);
    if (cross(a-c,b-c) * cross(a-d,b-d) < 0 &&    // strictly intersect.
        cross(c-a,d-a) * cross(c-b,d-b) < 0) return true;
    if (cross(a-c,b-c) == 0 && dot(a-c,b-c) < 0) return true;
    if (imag(c) <= imag(m) && imag(m) < imag(d))  // strictly contain.
      if (cross(c-m,d-m) < 0) in = !in;
    if (cross(c-m,d-m) == 0 && dot(c-m,d-m) <= EPS) on = true;
  }
  return !on && in;
}

//面積
double area(const vector<point>& p) {
	double A = 0;
	for (int i = 0; i < p.size(); ++i)
		A += cross(curr(p, i), next(p, i));
	return A / 2.;
}

//凸包判定
bool isConvex(vector<point> poly){
	int sz = poly.size();
	REP(i,sz){
		if(ccw(poly[i],poly[(i+1)%sz],poly[(i+2)%sz]) == -1)return false;
	}
	return true;
}

double convex_diameter(const vector<point> &pt) {
  const int n = pt.size();
  int is = 0, js = 0;
  for (int i = 1; i < n; ++i) {
    if (imag(pt[i]) > imag(pt[is])) is = i;
    if (imag(pt[i]) < imag(pt[js])) js = i;
  }
  double maxd = norm(pt[is]-pt[js]);

  int i, maxi, j, maxj;
  i = maxi = is;
  j = maxj = js;
  do {
    if (cross(diff(pt,i), diff(pt,j)) >= 0) j = (j+1) % n;
    else i = (i+1) % n;
    if (norm(pt[i]-pt[j]) > maxd) {
      maxd = norm(pt[i]-pt[j]);
      maxi = i; maxj = j;
    }
  } while (i != is || j != js);
  return sqrt(maxd); /* farthest pair is (maxi, maxj). */
}

vector<point> convex_cut(const vector<point> P, const segment& l) {
  vector<point> Q;
  for (int i = 0; i < P.size(); ++i) {
    point A = curr(P, i), B = next(P, i);
    if (ccw(l[0], l[1], A) != -1) Q.push_back(A);
    if (ccw(l[0], l[1], A)*ccw(l[0], l[1], B) < 0)
      Q.push_back(crosspointSS(segment(A, B), l));
  }
  return Q;
}

int main(){
	cin.tie(0);cout.tie(0);ios::sync_with_stdio(false);

	return 0;
}
\end{lstlisting}


\section{Number Theory}
\subsection{Extend Gcd}
\begin{lstlisting}
// ax + by = gcd(a, b)
long long extGcd(long long a, long long b, long long &x, long long &y) {
    if(b == 0) {
        x = 1;
        y = 0;
        return a;
    }
    long long d = extGcd(b, a%b, y, x);
    y -= a / b * x;
    return d;
}

// aで割った余りがx、bで割った余りがyの時の、abで割った余り
// aとbは互いに素
long long CRT(long long a, long long b, long long x, long long y) {
    long long c, d;
    extGcd(a, b, c, b);
    ll ret = (a * y % (a * b) * c % (a * b) + (a * b)) % (a * b);
    return (ret + b * x  % (a * b)* d % (a * b) + (a * b)) % (a * b);
}


int main(){

    ll x, y;
    ll out = extGcd(111, 30, x , y);
    cout << x << " " << y << " " << 111 * x + 30 * y << " " << out << endl;

	return 0;
}
\end{lstlisting}

\subsection{Baby Step Giant Step}
\begin{lstlisting}
//g^x = y(mod MOD)を満たすxを探す
long long babyStepGiantStep(ll g, ll y) {
	long long m = sqrt(MOD) + 1;
	map<long long,long long> mp;
	long long tmp = 1;
	for(int i = 0;i < m;i++){
		mp[tmp] = i;
		tmp = (tmp * g) % MOD;
	}
	long long gm = 1;
	for(int i = 0;i < m;i++)gm = (gm * g) % MOD;
	long long factg = mod_div(1, gm);
	long long ganma = y;
	for(int i = 0;i < m;i++){
		auto it = mp.find(ganma);
		if(it != mp.end()){
			return i * m + (it->second);
		}
		ganma = ganma * factg % MOD;
	}
	return -1;
}
\end{lstlisting}


\section{Other}
\subsection{MOD}
\begin{lstlisting}
#define MOD 1000000007

long long totient(long long a) {
	ll ret = a;
	for(int i = 2;i * i <= a;i++){
		if(a % i == 0){
			ret = ret * (i - 1) / i;
			while(a % i == 0)a /= i;
		}
	}
	if(a > 1)ret = ret * (a - 1) / a;
	return ret;
}

// a / b (mod p) b と p は互いに素
// O(sqrt(p))かかるので注意
long long totient_div(long long a, long long b, long long p) {
	long long ret = a;
	long long tp = totient(p) - 1;
	long long c = b;
	for(int i = 0;i <= 60;i++){
		if(tp & (1LL << i))ret = (ret * c) % p;
		c = (c * c) % p;
	}
	return ret;
}

//a^b % MOD
long long mod_pow(ll a,ll b){
	long long ret = 1;
	long long c = a;
	for(int i = 0;i <= 60;i++){
		if(b & (1LL << i))ret = (ret * c) % MOD;
		c = (c * c) % MOD;
	}
	return ret;
}

//aをbで割る
long long mod_div(long long a,long long b){
	long long tmp = MOD - 2,c = b,ret = 1;
	while(tmp > 0){
		if(tmp & 1){
			ret *= c;ret %= MOD;
		}
		c *= c;c %= MOD;tmp >>= 1;
	}
	return a*ret%MOD;
}

#define MAX_K 333333
vector<long long> kaijo(MAX_K);
long long combination(long long n, long long r){
    if(n < r || n < 0 || r < 0)
        return 0;
	if(kaijo[0] != 1){
		kaijo[0] = 1;
		for(long long i = 1;i < MAX_K;i++)kaijo[i] = (kaijo[i-1] * i) % MOD;
	}
    long long ret = kaijo[n];
	long long tmp = (kaijo[r] * kaijo[n-r]) % MOD;
    return mod_div(ret,tmp);
}
\end{lstlisting}

\subsection{Gaussian elimination}
\begin{lstlisting}
#define BIT_N 2002
typedef bitset<BIT_N> Bit;

int Gaussian_elimination(vector<Bit> &v){
    int rank = 0;
    REP(i, BIT_N) {
        int pla = -1;
        for(int j = rank;j < v.size();j++){
            if(v[j][i] == 1){
                pla = j;
                break;
            }
        }

        if(pla != -1){
            swap(v[rank], v[pla]);
            for(int j = 0;j < v.size();j++){
                if(j == rank)continue;
                if(v[j][i] == 1)v[j] ^= v[rank];
            }
            rank++;
        }
    }
    return rank;
}

int Gaussian_elimination(vector<ll> &v){
	int rank = 0;
	REP(i, 60){
		int pla = -1;
		for(int j = rank;j < v.size();j++){
			if(v[j] & (1LL << i)){
				pla = j;
				break;
			}
		}
		if(pla != -1){
			swap(v[rank], v[pla]);
			REP(j,v.size()){
				if(j == rank)continue;
				if(v[j] & (1LL << i))v[j] ^= v[rank];
			}
			rank++;
		}
	}
    return rank;
}
\end{lstlisting}

\subsection{Levenshtein}
\begin{lstlisting}
#define D 10000
int dp[D][D];
void LS(string a,string b)
{
	REP(i,a.size()+1)dp[i][0]=i;
	REP(i,b.size()+1)dp[0][i]=i;
	for(int i=1;i<=a.size();i++)
	{
		for(int j=1;j<=b.size();j++)
		{
			int q=1;if(a[i-1]==b[j-1])q = 0;
			dp[i][j]=min(min(dp[i-1][j-1]+q,dp[i][j-1]+1),dp[i-1][j]+1);
		}
	}
}
int main()
{
	string a,b;cin>>a>>b;
	cout<<(LS(a,b),dp[a.size()][b.size()])<<endl;
	return 0;
}
\end{lstlisting}

\subsection{Independent Set}
\begin{lstlisting}
//0 start
//40頂点までの最大独立点集合を求める
struct IndependentSet {

	vector<pair<int,int> > v;
	vector<int> dp;
	vector<int> okaa;
	vector<int> okbb;
	vector<int> okba;
	int n;

	IndependentSet(int _n):dp(1<<(_n+1)/2),okaa((_n+1)/2),okbb((_n+1)/2),okba((_n+1)/2){
		n = _n;
	}

	void addEdge(int a,int b){
		v.EB(a, b);
	}

	int solve(){
		int Sa = n/2;
		int Sb = n - Sa;
		//initialize
		REP(i,Sa){
			okaa[i] = (1 << Sa) - 1;
		}
		REP(i,Sb){
			okbb[i] = (1 << Sb) - 1;
			okba[i] = (1 << Sa) - 1;
		}

		REP(q,v.size()){
			int a = min(v[q].FI,v[q].SE);
			int b = max(v[q].FI,v[q].SE);
			if(b < Sa){
				okaa[a] &= ~(1 << b);
				okaa[b] &= ~(1 << a);
			}
			else if(a < Sa){
				okba[b-Sa] &= ~(1 << a);
			}
			else{
				okbb[a-Sa] &= ~(1 << (b-Sa));
				okbb[b-Sa] &= ~(1 << (a-Sa));
			}
		}

		REP(q,(1<<Sa)){
			int now = q;
			int num = 0;
			REP(i,Sa){
				if(q & (1 << i)){now &= okaa[i];num++;}
			}
			if(now != q)continue;
			dp[q] = num;
		}

		REP(i,(1<<Sa)){
			REP(j,Sa){
				dp[i|(1<<j)] = max(dp[i],dp[i|(1<<j)]);
			}
		}

		int ret = 0;
		REP(q,(1<<Sb)){
			int node = q;
			int now = 0;
			REP(i,Sb){
				if(q & (1<<i)){node &= okbb[i];now++;}
			}
			if(node!=q)continue;

			int aa = (1<<Sa)-1;
			REP(i,Sb)
			{
				if(q & (1<<i))aa &= okba[i];
			}
			ret = max(ret,now+dp[aa]);
		}

		return ret;
	}

};

int main(){
	int n,m;
	cin >> n >> m;
	IndependentSet seica(n);
	REP(i,m){
		int a,b;cin >> a >> b;
		seica.addEdge(a-1, b-1);
	}

	cout << seica.solve() << endl;

	return 0;
}
\end{lstlisting}

\subsection{Matrix}
\begin{lstlisting}
template<class type>
class matrix {
private:
	size_t H, W;
	std::vector<type> mat;
public:
	matrix() : H(0), W(0), mat(std::vector<type>()) {};
	matrix(size_t N_) : H(N_), W(N_), mat(std::vector<type>(N_ * N_)) {};
	matrix(size_t H_, size_t W_) : H(H_), W(W_), mat(std::vector<type>(H_ * W_)) {};
	type& at(size_t r, size_t c) { return mat[r * W + c]; }
	type at(size_t r, size_t c) const { return mat[r * W + c]; }
	matrix& operator+=(const matrix& m) {
		assert(H == m.H && W == m.W);
		for (size_t i = 0; i < H; ++i) {
			for (size_t j = 0; j < W; ++j) {
				mat[i * W + j] += m.mat[i * W + j];
			}
		}
	}
	matrix& operator-=(const matrix& m) {
		assert(H == m.H && W == m.W);
		for (size_t i = 0; i < H; ++i) {
			for (size_t j = 0; j < W; ++j) {
				mat[i * W + j] -= m.mat[i * W + j];
			}
		}
	}
	matrix& operator*=(const matrix& m) {
		assert(W == m.H);
		matrix ans(H, m.W);
		for (size_t i = 0; i < H; ++i) {
			for (size_t j = 0; j < m.W; ++j) {
				for (size_t k = 0; k < W; ++k) {
					ans.mat[i * m.W + j] += mat[i * W + k] * m.mat[k * m.W + j];
				}
			}
		}
		return (*this) = ans;
	}
	matrix operator+(const matrix& m) const { return matrix(*this) += m; }
	matrix operator-(const matrix& m) const { return matrix(*this) -= m; }
	matrix operator*(const matrix& m) const { return matrix(*this) *= m; }
};

template<class type>
matrix<type> unit(int N) {
	matrix<type> ret(N);
	for (size_t i = 0; i < N; ++i) {
		ret.at(i, i) = type(1);
	}
	return ret;
}
\end{lstlisting}

\subsection{Fast Fourier Transform}
\begin{lstlisting}
struct FastFourierTransform {
  using C = complex< double >;

  const double PI = acos(-1);
  vector< vector< C > > rts, rrts;

  void ensure_base(int N) {
    if(rts.size() >= N) return;
    rts.resize(N), rrts.resize(N);
    for(int i = 1; i < N; i <<= 1) {
      if(rts[i].size()) continue;
      rts[i].resize(i), rrts[i].resize(i);
      for(int k = 0; k < i; k++) {
        rts[i][k] = polar(1.0, PI / i * k);
        rrts[i][k] = polar(1.0, -PI / i * k);
      }
    }
  }

  void DiscreteFourierTransform(vector< C > &F, bool rev) {
    const int N = (int) F.size();
    auto &r = rev ? rrts : rts;

    for(int i = 0, j = 1; j + 1 < N; j++) {
      for(int k = N >> 1; k > (i ^= k); k >>= 1);
      if(i > j) swap(F[i], F[j]);
    }
    ensure_base(N);
    C s, t;
    for(int i = 1; i < N; i <<= 1) {
      for(int j = 0; j < N; j += i * 2) {
        for(int k = 0; k < i; k++) {
          s = F[j + k];
          t = C(F[j + k + i].real() * r[i][k].real() - F[j + k + i].imag() * r[i][k].imag(),
                F[j + k + i].real() * r[i][k].imag() + F[j + k + i].imag() * r[i][k].real());
          F[j + k] = s + t, F[j + k + i] = s - t;
        }
      }
    }
    if(rev) for(int i = 0; i < N; i++) F[i] /= N;
  }

  vector< long long > Multiply(const vector< int > &A, const vector< int > &B) {
    int sz = 1;
    while(sz < A.size() + B.size() - 1) sz <<= 1;
    vector< C > F(sz), G(sz);
    for(int i = 0; i < A.size(); i++) F[i] = A[i];
    for(int i = 0; i < B.size(); i++) G[i] = B[i];
    DiscreteFourierTransform(F, false);
    DiscreteFourierTransform(G, false);
    for(int i = 0; i < sz; i++) F[i] *= G[i];
    DiscreteFourierTransform(F, true);
    vector< long long > X(A.size() + B.size() - 1);
    for(int i = 0; i < A.size() + B.size() - 1; i++) X[i] = F[i].real() + 0.5;
    return (X);
  }
};


int main() {
  int N;
  scanf("%d", &N);
  vector< int > A(N + 1), B(N + 1);
  for(int i = 0; i < N; i++) scanf("%d %d", &A[i + 1], &B[i + 1]);
  FastFourierTransform fft;
  auto C = fft.Multiply(A, B);
  for(int i = 1; i <= 2 * N; i++) printf("%lld\n", C[i]);
}
\end{lstlisting}


\end{document}
